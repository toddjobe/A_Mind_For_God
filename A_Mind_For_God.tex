\documentclass{tufte-handout}

%\geometry{showframe}% for debugging purposes -- displays the margins

\usepackage{amsmath}

% Set up the images/graphics package
\usepackage{graphicx}
\setkeys{Gin}{width=\linewidth,totalheight=\textheight,keepaspectratio}
\graphicspath{{graphics/}}

\title{A Mind for God}
\author[R. Todd Jobe]{R. Todd Jobe}
\date{31 March 2013}  % if the \date{} command is left out, the current date will be used

% The following package makes prettier tables.  We're all about the bling!
\usepackage{booktabs}

% The units package provides nice, non-stacked fractions and better spacing
% for units.
\usepackage{units}

% The fancyvrb package lets us customize the formatting of verbatim
% environments.  We use a slightly smaller font.
\usepackage{fancyvrb}
\fvset{fontsize=\normalsize}

% Small sections of multiple columns
\usepackage{multicol}

% Provides paragraphs of dummy text
\usepackage{lipsum}

% These commands are used to pretty-print LaTeX commands
\newcommand{\doccmd}[1]{\texttt{\textbackslash#1}}% command name -- adds backslash automatically
\newcommand{\docopt}[1]{\ensuremath{\langle}\textrm{\textit{#1}}\ensuremath{\rangle}}% optional command argument
\newcommand{\docarg}[1]{\textrm{\textit{#1}}}% (required) command argument
\newenvironment{docspec}{\begin{quote}\noindent}{\end{quote}}% command specification environment
\newcommand{\docenv}[1]{\textsf{#1}}% environment name
\newcommand{\docpkg}[1]{\texttt{#1}}% package name
\newcommand{\doccls}[1]{\texttt{#1}}% document class name
\newcommand{\docclsopt}[1]{\texttt{#1}}% document class option name

\begin{document}

\maketitle% this prints the handout title, author, and date

\begin{abstract}
\noindent Paul told the Corinthians that, "...not many of you were wise according to wordly standards...,"\footnote{I Cor. 1:26} when they were called to be part of the Church. Is this statement true of Churches today?  Today's sermon examines this question and addresses the apparent rift we see between intellectuals and Christians.
\end{abstract}

%\printclassoptions
\section{Introduction}\label{sec:introduction}
\newthought{There was some hint that this had to do with the factions.}  Each faction thought they had the wisest teacher? And, people seemed to be following teachers who had lifted themselves up.  Paul makes it clear that he came to preach the gospel, and not with eloquent wisdom.\footnote{"For Christ did not send me to baptize but to preach the gospel, and not with words of eloquent wisdom, lest the cross of Christ be emptied of its power.  For the word of the cross is folly to those who are perishing, but to us who are being saved it is the power of God." I Cor. 1:17-18} That is the gospel, when rightly presented is accessible not difficult, centered on Christ's message not the wise musings of a man.  And, that simplicity turns some people off.  He says that not many wise, not many powerful, not many of noble birth were Christians at Corinth.  They should have been happy about that, yet it seems that one of the characteristics of the factions that existed among them was that tried to generate appeal according to the world's wisdom.  It seems that this is still true today.  We aren't generally made up of societal elites, CEOs, professors.  That's not to say that there aren't churches or individuals who fall into those categories, but in general the church is hopefully made of a broad cross-section of society.  

And, in fact we bristle against the elite, particularly in regard to wisdom.  More than ever it seems as though there is a significant rift between the university environment, the intellectual elite and the church.  Christians, concerned about their children and not without cause, see the college professor as the enemy, an influence that is going to try and pull your son or daughter away from the faith.  And, I can name person after person that I grew up with who were faithful up to their mid-20s and have subsequently fallen away.  Some of those I would say, are directly attributable to the influence of the world's wisdom.

While that may be my personal experience, there are other lines of evidence that support this conclusion.  
If you are believe in the inerrancy of the bible you are less likely to have higher education.
Top scientists are much less likely to believe in God.

My goal today is to lay before some observations that I have made about the reason why it is diffcult for the highly educated to maintain their faithe or to be converted to become followers of Christ.  I'm going to address 4 topics.  You could certainly come up with others, and in no way do I claim to exhast the possibile reasons one might reject Christ.  These are simply 4 ones that I have personally observed.  We're going to address these topics headon and give a Bible perspective on each of them.  We're going to talk about how to avoid falling into these traps.  We're also going to point the finger back at ourselve, because the possibility of falling into these traps is not limited to the godless.  We as Christians can be or are being influences by them.  Hopefully this treatement will move us from seeing the highly intellectal not as the enemy but simply one of the world. 

\section{The Gospel Is Simple, Not Complicated}\label{sec:invented-complexity}
The world is complicated.  Statistics as a field exists solely to help people deal with the complexities of the real world and to help them draw rational conclusions from evidence that is riddled with uncertainty. But even with good analysis of data, there's usually more than one persective that may be right.  So, one might find it difficult to accept that in a world of uncertainty that absolute truth exists.  And, what's more, that that abosolute truth may be found in the Word of God.  Pilot found this difficult, too.\footnote{"Then Pilate said to him, 'So you are a king?' Jesus answered, 'You say that I am a king. For this purpose I was born and for this purpose I have come into the world---to bear witness to the truth.  Everyone who is of the truth listens to my voice.' Pilate said to him, 'What is truth?' John18:37-38} The agnostic would say that absolute truth might exist but that it is too complicated for any human to figure out.  In the Old Testament Naaman's wise servant cautioned him against wishing for a complicated answer when a simple answer lay right in front of him.\footnote{I Kings 5:1:14}.  Paul reminds the Corinthians of the same thing: the gospel is simply Christ crucified.\footnote{"For since, in the wisdom of God, the worlld did not know God through wisdom, it pleased God through the folly of what we preach to save those who believe.  For Jews demand signs and Greeks seek wisdom, but wee preach Christ crucified, a stumbling block to the Jews and folly to the Gentiles..." I Corinthians 1:21-23} It shows God's wisdom that the gospel is so simple.  Then all have access to it, not a select few who 'get it'.\footnote{In fact, making the gospel complicated is one way that evil men have tried since the beginning of the church to keep 'undesireables' out.  See I John.} Let the simple answer of the gospel suffice. Then, we all have a chance at heaven.  

\section{The Bible Is About Answers, Not Questions}\label{sec:questions-instead-of-answers}
Questions Instead of Answers
The bible is about answers and not questions
Deuteronomy 10:12-14 God tells you what to do and you do it
'Greeks alwasy searching but unable to come to a knowledge of the truth.'

\section{God's Works Are Supernatural, Not Natural}
Rich man and Lazarus 'But if someone came back from the dead'
Miracles are supposed to be impossible
Don't spend all your time trying to explain miracles.  Unless you were there, you wouldn't know that a miracle had taken place.  That's the completness of the miracles of God.  It was as if the problem had never been.
When in doubt, let your mind focus on one miracle and one miracle only.  Jesus is not dead.  And, if Jesus is not dead, then all the rest of the miracles are true.

\section{God's Thoughts Are Supernatural, Not Natural}
My thoughts are higher than your thoughts.
Why doesn't God just save everyone?  If he's all powerful, then why doens't he just 
What makes you think you know how to run the universe?
Job had this problem, he spent the entire book saying that if only he could get an audience with God then he would show God that he, Job, was in the right and tthat his suffering was injust.  Eventually, he does get an audience with God, but the converation is sort of one sided.  God says to Job, were you there when I did . He says, "Do you know about this?" And, at the end of God tell Job all the reasons why he is infinitely greater than Job, Job can only say, "I repent in dust an ashes."
Romans 9:20 " But who are you, O man, to answer back to God? Will what is molded say to the molder "Why have you made me like this?"

\section{God Gives To You, But Also Asks Of You} 
Where your treasure lies, there will your heart be also.
If you were say to me, "Todd conjugate Gaussian unknown precision, gamma."
My first 34 years of life were spent doing nothin but learning new stuff.  And even today my job is mostly me studying about new things and trying to apply the things i learn to solve problems.  And, I'm not unique.  In academic circles, people spend their whole lives learning something new.  And, there's any number of technical questios you could ask some professional that they'd be able to answer off the top of their head, because they've been concientious at their job.  But, if I were to ask that same person, "What's the main point of Romans Chapter 4?"  Who would be able to say, immediately, "The righteous are justified by their faith."?
Ezra 7:10 "For Ezra had set his heart to study the Law of the Lord and to do it and to teach His statutes and rules in Israel.

Pride
trust your own instincts above all else

\end{document}
